%%%%%%%%%%%%%%%%%
% Date commands %
%%%%%%%%%%%%%%%%%
\newdateformat{experienceDate}{\monthname[\THEMONTH] \THEYEAR}

%% Text used if no "experienceEnd" is specified
\newcommand\currentJobEndDate{%
	\french{actuel}%
	\english{Current}%
}

%% Display dates based on presence and content of experienceStart and experienceEnd
\newcommand\experienceSpan{
	\experienceDate\displaydate{\replacabledatename{experienceStart}} - %
	\ifnum \getdateyear{\replacabledatename{experienceEnd}}=1%
		\currentJobEndDate%
	\else%
		\experienceDate\displaydate{\replacabledatename{experienceEnd}}%
	\fi%
}

%%%%%%%%%%%%%%
% Experience %
%%%%%%%%%%%%%%
%% Define the experience start and end days
%% Format is {day}{month}{year}
%% An experienceEnd set to 1/1/1 represents a current job.
%\newreplacabledate{experienceStart}{1}{10}{2010}
%\newreplacabledate{experienceEnd}{1}{1}{1}

%% CV entry (one for each job and each language)
%% Format is {job description}{company}{City}{Country}
%\english{
%\cventry{\experienceSpan}{Java/JEE/Spring developer/trainer}{Zenika}{Paris}{France}{
%% content
%}
%}

%%%%%%%%%%%%%%%%%%%%%%%%
% Atlassian %
%%%%%%%%%%%%%%%%%%%%%%%%
\newreplacabledate{experienceStart}{1}{9}{2013}
\newreplacabledate{experienceEnd}{1}{3}{2014}

\english{
\cventry{\experienceSpan}{Java developer}{Build Engineering, Atlassian}{Sydney}{Australia}{
	\begin{itemize}
		\item Development and maintenance of the Continuous Integration platform
		\begin{itemize}
			\item Development of plugins for Bamboo, the CI platform
            \item Modular project based on Spring, using multiple web frameworks (Struts, Velocity, Wicket, JSF)
            \item Involvement in the Sakai community and contribution to the main project
		\end{itemize}
	\end{itemize}
}
}

%%%%%%%%%%%%%%%%%%%%%%%%
% University of Oxford %
%%%%%%%%%%%%%%%%%%%%%%%%
\newreplacabledate{experienceStart}{9}{1}{2012}
\newreplacabledate{experienceEnd}{31}{7}{2013}

\english{
\cventry{\experienceSpan}{Java/Sakai developer}{IT Services, University of Oxford}{Oxford}{United Kingdom}{
	\begin{itemize}
		\item Development of Weblearn, University of Oxford's VLE
		\begin{itemize}
			\item Customisation of an instance of the Open-Source project Sakai for the needs of the university
      \item Modular project based on Spring, using multiple web frameworks (Struts, Velocity, Wicket, JSF)
      \item Involvement in the Sakai community and contribution to the main project
		\end{itemize}
		\item Improvement of the Sakai search engine
		\begin{itemize}
			\item Implementation of the Sakai-Search API using Solr as the back-end search engine
      \item Rewriting of the Sakai-Search API in the project Search2, improving separation between API and implementation to simplify the creation of future implementations
		\end{itemize}
%		\item Creation of tools allowing the consumption of Sakai APIs by third-party applications
%		\begin{itemize}
%      \item Introduction of OAuth1.0 to Sakai, allowing connections on the behalf of any user
%      \item Improvement of the existing services and creation of new web services
%		\end{itemize}
    \item Reengineering of the anti-plagiarism integration within Sakai
    \begin{itemize}
      \item Creation of a Turnitin library, allowing to make calls to the Turnitin API
      \item Rewriting of the assignment review system to use the Turnitin library
		\end{itemize}
	\end{itemize}
}
}

\french{
\cventry{\experienceSpan}{Développeur Java/Sakai}{Université d'Oxford, Services IT}{Oxford}{Royaume-Uni}{
	\begin{itemize}
		\item Dévelopemment du project Weblearn, VLE de l'Université d'Oxford
		\begin{itemize}
			\item Adaptation du projet Open-Source Sakai aux besoins de l'université
      \item Projet basé sur Spring, reposant sur divers frameworks web (Struts, Velocity, Wicket, JSF) du à la modularité du projet
      \item Implication dans la communauté de Sakai et contribution régulière au projet
		\end{itemize}
		\item Amélioration du moteur de recherche de Sakai
		\begin{itemize}
			\item Création d'une implementation de Sakai-Search basée sur Solr comme principal moteur de recherche
      \item Réécriture de l'API Sakai-Search sous le nouveau projet Search2, promouvant une séparation claire entre API et implémentation pour simplifier la création d'autres implémentations
		\end{itemize}
%		\item Création d'outils permettant l'accès aux API de Sakai par des applications tierces
%		\begin{itemize}
%      \item Ajout du support d'OAuth1.0 à Sakai, permettant d'autres applications de se connecter en tant qu'utilisateurs
%      \item Amélioration de services existants et création de nouveaux services REST
%		\end{itemize}
    \item Réécriture de l'intégration des outils anti-plagia dans Sakai
    \begin{itemize}
      \item Création d'une librairie Turnitin permettant de d'appeller l'API de Turnitin
      \item Réécriture du système d'analyse de documents, intégration de la librairie Turnitin
		\end{itemize}
	\end{itemize}
}
}


%%%%%%%%%%
% Zenika %
%%%%%%%%%%
\newreplacabledate{experienceStart}{1}{10}{2010}
\newreplacabledate{experienceEnd}{1}{11}{2011}

\english{
\cventry{\experienceSpan}{Java/JEE/Spring consultant and trainer}{Zenika}{Paris}{France}{
	\begin{itemize}
		\item Work and study of Spring technologies:
		\begin{itemize}
			\item Spring IOC, Spring AOP, Spring DAO
			\item Introduction to RabbitMQ, an AMQP broker, and publication of blog posts on its administration and usage: \httplink[Blog Zenika AMQP (fr)]{blog.zenika.com/index.php?tag/amqp}
		\end{itemize}
%		\item Participation to the USI challenge: \httplink[Octo - Challenge USI 2011]{sites.google.com/a/octo.com/challengeusi2011/}
%		\begin{itemize}
%			\item Development of an application able to support the load of one million users simultaneously connected, and a total of one billion users registered
%			\item Application using long polling and a data-grid, based on the SpringSource/VMware vFabric stack
%		\end{itemize}
%		\item One of the first developers of DORM, an european projet used to manage dependencies and packages:
%		\begin{itemize}
%			\item Integrated CUDF to handle dependencies conflicts.
%			\item Created a metadata format adapted to dependency management.
%			\item Established a connection between multiple repositories to enable mirroring and improve performances.
%		\end{itemize}
		\item Gave technical trainings to professionals, on various subjects:
		\begin{itemize}
			\item Writing and teaching of an introductory course to XML and its use in Java
			\item Creation and instruction of a course on Version Control Systems, specifically git
%			\item Fixed and updated a Struts2 course
			\item Training for professionals on the Java programming language and basics
		\end{itemize}
	\end{itemize}
}
}

\french{
\cventry{\experienceSpan}{Consultant et formateur Java/JEE/Spring}{Zenika}{Paris}{France}{
	\begin{itemize}
		\item Utilisation des technologies Spring :
		\begin{itemize}
			\item Spring IOC, Spring AOP, Spring DAO
			\item Étude d'un broker AMQP, RabbitMQ, et publication de blogposts sur son administration et fonctionnement : \httplink[Blog Zenika AMQP (fr)]{blog.zenika.com/index.php?tag/amqp}
		\end{itemize}
%		\item Participation au concours USI : \httplink[Octo - Concours USI 2011 (fr)]{sites.google.com/a/octo.com/challengeusi2011/}
%		\begin{itemize}
%			\item Création d'une application devant supporter la charge d'un million d'utilisateurs connectés simultanément et un total d'un milliard d'utilisateurs inscrits
%			\item Application utilisant du long-polling et une data-grid, entièrement basée sur la pile applicative SpringSource/VMWare vFabric
%		\end{itemize}
%		\item Participation au développement de DORM, projet européen de gestion de paquets et dépendances :
%		\begin{itemize}
%			\item Integration du format CUDF pour la gestion des conflits de dépendances.
%			\item Création d'un format de métadata adapté à la gestion de dépendances.
%			\item Mise en place d'une connexion entre plusieurs dépôts pour permettre la réplication et améliorer les performances.
%		\end{itemize}
		\item Formation de professionnels sur des sujets techniques variés :
		\begin{itemize}
			\item Écriture et enseignement d'une formation XML orientée Java
			\item Création et instruction d'une formation au système de contrôle de version git
%			\item Correction et mise en place d'un cours pour Struts2
			\item Formations sur l'usage de Java SE
		\end{itemize}
	\end{itemize}
}
}


%%%%%%%%%%%
% Supinfo %
%%%%%%%%%%%
\newreplacabledate{experienceStart}{1}{9}{2008}
\newreplacabledate{experienceEnd}{1}{9}{2011}

\english{
\cventry{\experienceSpan}{Supinfo Sun Certified Trainer}{SUPINFO - The International Institute of Information Technology}{}{France}{
	Teach Java/JEE to MSc IT students (up to 80 students per lecture)
	\begin{itemize}
		\item Preparation of second year students for SCJA and SCJP certifications:
		\begin{itemize}
			\item First steps in Java - IDE, Tools, Object Oriented Programming, Design patterns
			\item Java SE - Syntax, Collection, Generics, Threads, I/O, Sockets, JDBC, Swing, Look and Feel
		\end{itemize}
		\item Preparation of third year students for SCWCD and SCMAD certifications:
		\begin{itemize}
			\item Java and the web - Tomcat, Servlets, JSP, JSTL, Struts 1.2
			\item Java Mobile - Java ME principles, CLDC, MIDP
		\end{itemize}
		\item Preparation of fourth year students for the SCBCD certification:
		\begin{itemize}
			\item Service oriented architectures - JEE, Web services, ORM, JPA, Glassfish, JMS
			\item Web applications - JSF, Richfaces, Facelets
		\end{itemize}
	\end{itemize}
}
}

\french{
\cventry{\experienceSpan}{Supinfo Sun Certified Trainer}{SUPINFO - The International Institute of Information Technology}{}{France}{
	Formateur Java/JEE pour des étudiants de master (jusqu'à 80 étudiants par cours).
	\begin{itemize}
		\item Formation des deuxième années (L2) aux certifications SCJA et SCJP :
		\begin{itemize}
			\item Premiers pas en Java - IDE, Tools, Object Oriented Programming, Design patterns
			\item Java SE - Syntaxe, Collection, Generics, Threads, I/O, Sockets, JDBC, Swing, Look and Feel
		\end{itemize}
		\item Formation des troisième années (L3) aux certifications SCWCD et SCMAD :
		\begin{itemize}
			\item Java et le web - Tomcat, Servlets, JSP, JSTL, Struts 1.2
			\item Java Mobile - les bases de Java ME, CLDC, MIDP
		\end{itemize}
		\item Formation des quatrième années (M1) à la certification SCBCD :
		\begin{itemize}
			\item Architectures orientées service - JEE, Web services, ORM, JPA, Glassfish, JMS
			\item Applications web - JSF, Richfaces, Facelets
		\end{itemize}
	\end{itemize}
}
}


%%%%%%%%
% Bull %
%%%%%%%%
\newreplacabledate{experienceStart}{1}{7}{2009}
\newreplacabledate{experienceEnd}{1}{10}{2010}

\english{
\cventry{\experienceSpan}{Java/JEE Developer}{Bull SAS}{Marseille}{France}{
	City of Marseilles' project - Integration of an unified financial management solution for the city of Marseilles {\em(4 months)}
	\begin{itemize}
		\item Deployment of a SSO across each component of the solution
		\begin{itemize}
			\item Set up of Central Authentication Service server (CAS, Jasig)
			\item Creation of a customized connector to LDAP and a MySQL database
%			\item Customisation of GateIn (Exo), Alfresco and Coriolis (Bull) to allow login through the SSO
		\end{itemize}
	\end{itemize}
	Lefebvre's project - Development of a management platform for Congés Spectacle users and clients {\em(1 year and 4 months)}
	\begin{itemize}
		\item Creation and integration of an additional module to OpenIris Finance (Lefebvre Software) to manage of members, their contributions and their paid leaves
		\begin{itemize}
			\item Application conceptualised with Enterprise Architect (Sparx Systems)
			\item Based on a combination of EJB \& Hibernate and JSP \& Struts
		\end{itemize}
	\end{itemize}
}
}

\french{
\cventry{\experienceSpan}{Développeur  Java/JEE}{Bull SAS}{Marseille}{France}{
	Projet ville de Marseille - Integration d'une solution de gestion financière unifiée pour la ville de Marseille {\em(4 mois)}
	\begin{itemize}
		\item Étude et mise en place d'un système d'authentification unique au travers des différents composants de la solution.
		\begin{itemize}
			\item Installation d'un Central Authentication Service server (CAS, Jasig)
			\item Création d'un connecteur personnalisé vers du LDAP et une base de données MyQSL
%			\item Intégration du système de connection dans GateIn (Exo), Alfresco et Coriolis (Bull)
		\end{itemize}
	\end{itemize}
	Projet Lefebvre - Développement d'une plateforme de gestion des adhérents de Congés Spectacles {\em(1 an et 4 mois)}
	\begin{itemize}
		\item Création et intégration d'un module additionnel à OpenIris Finance (Lefebvre Software) pour ajouter la gestion des adhérents, de leurs cotisations et leurs congés.
		\begin{itemize}
			\item Conception de l'application - UML avec Enterprice Architect (Sparx Systems)
			\item Couches métier et vue - EJB \& Hibernate, JSP \& Struts
%			\item Processus de compilation et déploiement - Ant \& Ivy
		\end{itemize}
	\end{itemize}
}
}
