%%%%%%%%%%%%%%%%%
% Date commands %
%%%%%%%%%%%%%%%%%
\newdateformat{experienceDate}{\monthname[\THEMONTH] \THEYEAR}

%% Text used if no "experienceEnd" is specified
\newcommand\currentJobEndDate{%
	\french{actuel}%
	\english{Current}%
}

%% Display dates based on presence and content of experienceStart and experienceEnd
\newcommand\experienceSpan{
	\experienceDate\displaydate{\replacabledatename{experienceStart}} - %
	\ifnum \getdateyear{\replacabledatename{experienceEnd}}=1%
		\currentJobEndDate%
	\else%
		\experienceDate\displaydate{\replacabledatename{experienceEnd}}%
	\fi%
}

%%%%%%%%%%%%%%
% Experience %
%%%%%%%%%%%%%%
%% Define the experience start and end days
%% Format is {day}{month}{year}
%% An experienceEnd set to 1/1/1 represents a current job.
%\newreplacabledate{experienceStart}{1}{10}{2010}
%\newreplacabledate{experienceEnd}{1}{1}{1}

%% CV entry (one for each job and each language)
%% Format is {job description}{company}{City}{Country}
%\english{
%\cventry{\experienceSpan}{Java/JEE/Spring developer/trainer}{Zenika}{Paris}{France}{
%% content
%}
%}

%%%%%%%%%%
% Zenika %
%%%%%%%%%%
\newreplacabledate{experienceStart}{1}{10}{2010}
\newreplacabledate{experienceEnd}{1}{11}{2011}

\english{
\cventry{\experienceSpan}{Java/JEE/Spring developer/trainer}{Zenika}{Paris}{France}{
	\begin{itemize}
		\item Worked with Spring technologies:
		\begin{itemize}
			\item Spring IOC, Spring AOP, Spring DAO
			\item Studied of RabbitMQ, an AMQP broker, and publication of blog posts on its administration and usage: \httplink[Blog Zenika AMQP (fr)]{blog.zenika.com/index.php?tag/amqp}
		\end{itemize}
		\item Participated to the USI challenge: \httplink[Octo - Challenge USI 2011]{sites.google.com/a/octo.com/challengeusi2011/}
		\begin{itemize}
			\item Development of an application able to handle the load of one million users simultaneously connected, and a total of one billion users registered
			\item The application used long polling and a data-grid and was based on the SpringSource/VMware vFabric stack
		\end{itemize}
%		\item One of the first developers of DORM, an european projet used to manage dependencies and packages:
%		\begin{itemize}
%			\item Integrated CUDF to handle dependencies conflicts.
%			\item Created a metadata format adapted to dependency management.
%			\item Established a connection between multiple repositories to enable mirroring and improve performances.
%		\end{itemize}
		\item Gave technical trainings to professionals, on various subjects:
		\begin{itemize}
			\item Wrote and taught a course on XML basics and integration in Java
			\item Created and instructed a course on Version Control Systems, specifically git
%			\item Fixed and updated a Struts2 course
			\item Trained professionals to Java programming language and basics
		\end{itemize}
	\end{itemize}
}
}

\french{
\cventry{\experienceSpan}{Développeur et formateur Java/JEE/Spring}{Zenika}{Paris}{France}{
	\begin{itemize}
		\item Utilisation des technologies Spring :
		\begin{itemize}
			\item Spring IOC, Spring AOP, Spring DAO
			\item Étude d'un broker AMQP, RabbitMQ, et publication de blogposts sur son administration et fonctionnement : \httplink[Blog Zenika AMQP (fr)]{blog.zenika.com/index.php?tag/amqp}
		\end{itemize}
		\item Participation au concours USI : \httplink[Octo - Concours USI 2011 (fr)]{sites.google.com/a/octo.com/challengeusi2011/}
		\begin{itemize}
			\item Création d'une application devant supporter la charge d'un million d'utilisateurs connectés simultanément et un total d'un milliard d'utilisateurs inscrits
			\item Application utilisant du long-polling et une data-grid, entièrement basée sur la pile applicative SpringSource/VMWare vFabric
		\end{itemize}
%		\item Participation au développement de DORM, projet européen de gestion de paquets et dépendances :
%		\begin{itemize}
%			\item Integration du format CUDF pour la gestion des conflits de dépendances.
%			\item Création d'un format de métadata adapté à la gestion de dépendances.
%			\item Mise en place d'une connexion entre plusieurs dépôts pour permettre la réplication et améliorer les performances.
%		\end{itemize}
		\item Formation de professionnels sur des sujets techniques variés :
		\begin{itemize}
			\item Écriture et enseignement d'une formation XML orientée Java
			\item Création et instruction d'une formation au système de contrôle de version git
%			\item Correction et mise en place d'un cours pour Struts2
			\item Formations sur l'usage de Java SE
		\end{itemize}
	\end{itemize}
}
}


%%%%%%%%%%%
% Supinfo %
%%%%%%%%%%%
\newreplacabledate{experienceStart}{1}{9}{2008}
\newreplacabledate{experienceEnd}{1}{9}{2011}

\english{
\cventry{\experienceSpan}{Supinfo Sun Certified Trainer}{SUPINFO - The International Institute of Information Technology}{}{France}{
	Taught Java/JEE to MSc IT students (up to 80 students during a lecture)
	\begin{itemize}
		\item Trained second year students for SCJA and SCJP certifications:
		\begin{itemize}
			\item First steps in Java - IDE, Tools, Object Oriented Programming, Design patterns
			\item Java SE - Syntax, Collection, Generics, Threads, I/O, Sockets, JDBC, Swing, Look and Feel
		\end{itemize}
		\item Trained third year students for SCWCD and SCMAD certifications:
		\begin{itemize}
			\item Java and the web - Tomcat, Servlets, JSP, JSTL, Struts 1.2
			\item Java Mobile - Java ME principles, CLDC, MIDP
		\end{itemize}
		\item Trained forth year students for the SCBCD certification:
		\begin{itemize}
			\item Service oriented architectures - JEE, Web services, ORM, JPA, Glassfish, JMS
			\item Web applications - JSF, Richfaces, Facelets
		\end{itemize}
	\end{itemize}
}
}

\french{
\cventry{\experienceSpan}{Supinfo Sun Certified Trainer}{SUPINFO - The International Institute of Information Technology}{}{France}{
	Formateur Java/JEE pour des étudiants de master (jusqu'à 80 étudiants par cours).
	\begin{itemize}
		\item Formation des deuxième années (L2) aux certifications SCJA et SCJP :
		\begin{itemize}
			\item Premiers pas en Java - IDE, Tools, Object Oriented Programming, Design patterns
			\item Java SE - Syntaxe, Collection, Generics, Threads, I/O, Sockets, JDBC, Swing, Look and Feel
		\end{itemize}
		\item Formation des troisième années (L3) aux certifications SCWCD et SCMAD :
		\begin{itemize}
			\item Java et le web - Tomcat, Servlets, JSP, JSTL, Struts 1.2
			\item Java Mobile - les bases de Java ME, CLDC, MIDP
		\end{itemize}
		\item Formation des quatrième années (M1) à la certification SCBCD :
		\begin{itemize}
			\item Architectures orientées service - JEE, Web services, ORM, JPA, Glassfish, JMS
			\item Applications web - JSF, Richfaces, Facelets
		\end{itemize}
	\end{itemize}
}
}


%%%%%%%%
% Bull %
%%%%%%%%
\newreplacabledate{experienceStart}{1}{7}{2009}
\newreplacabledate{experienceEnd}{1}{10}{2010}

\english{
\cventry{\experienceSpan}{Java/JEE Developer}{Bull SAS}{Marseille}{France}{
	City of Marseilles' project - Integrated a unified financial management solution for the city of Marseilles {\em(4 months)}
	\begin{itemize}
		\item Established a unique authentication system across different components of the solution
		\begin{itemize}
			\item Set up a Central Authentication Service server (CAS, Jasig)
			\item Created a customized connector to LDAP and a MySQL database
			\item Modified GateIn (Exo), Alfresco and Coriolis (Bull) to allow login on the CAS
		\end{itemize}
	\end{itemize}
	Lefebvre's project - Developed a platform to manage members of Congés Spectacles {\em(1 year and 4 months)}
	\begin{itemize}
		\item Created and integrated an additional module to OpenIris Finance (Lefebvre Software) to add management of members, their contribution and their holidays
		\begin{itemize}
			\item Application conception - UML with Enterprise Architect (Sparx Systems)
			\item Business and views layers - EJB \& Hibernate, JSP \& Struts
			\item Build and deployment processes - Ant \& Ivy
		\end{itemize}
	\end{itemize}
}
}

\french{ 
\cventry{\experienceSpan}{Développeur  Java/JEE}{Bull SAS}{Marseille}{France}{
	Projet ville de Marseille - Integration d'une solution de gestion financière unifiée pour la ville de Marseille {\em(4 mois)}
	\begin{itemize}
		\item Étude et mise en place d'un système d'authentification unique au travers des différents composants de la solution.
		\begin{itemize}
			\item Installation d'un Central Authentication Service server (CAS, Jasig)
			\item Création d'un connecteur personnalisé vers du LDAP et une base de données MyQSL
			\item Intégration du système de connection dans GateIn (Exo), Alfresco et Coriolis (Bull)
		\end{itemize}
	\end{itemize}
	Projet Lefebvre - Développement d'une plateforme de gestion des adhérents de Congés Spectacles {\em(1 an et 4 mois)}
	\begin{itemize}
		\item Création et intégration d'un module additionnel à OpenIris Finance (Lefebvre Software) pour ajouter la gestion des adhérents, de leurs cotisations et leurs congés.
		\begin{itemize}
			\item Conception de l'application - UML avec Enterprice Architect (Sparx Systems)
			\item Couches métier et vue - EJB \& Hibernate, JSP \& Struts
			\item Processus de compilation et déploiement - Ant \& Ivy
		\end{itemize}
	\end{itemize}
}
}
